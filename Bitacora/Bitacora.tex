\documentclass{article}
\author{Simón Sloan García Villa}
\title{Retos durante el desarrollo de la aplicación de PHP}
\begin{document}
	\maketitle
	\newpage
	\tableofcontents
	\newpage
	\section{Introducción}
	Este documento permitirá mostrar las dificultades en las que como estudiante del curso de 
	\textbf{Arquitectura de Software}, tuve para desarrollar una aplicación sencilla para la inserción de clientes en una base de datos local \textbf{MySql}
	\section{Retos}
	\subsection{Entender la sintaxis basica de PHP}
	Este fue el primer proyecto de desarrollaba con \textbf{Php}, por lo cual tuve que destinar un tiempo importante a entender la sintaxis. Esto lo logré por medio del curso de Php de  \textbf{Código facilito}, el manual del propio Php y la inteligencia artificial. 
	\subsection{Investigar como hacer una conexión a Mysql desde PHP}
	En este apartado, miré varias opciones, entre las que tenemos principalmente PDO (PHP Data Objects), el cual no tomé porque ví la sintaxis y me pareció que podría ser más apropiada cuando ya tenga más tiempo manejando php. Por eso, usé la segunda opción que fue usar MySqli, que como se podrá intuir, era más sencilla y por lo tanto practica para este momento. 
	\subsection{Metodos de petición HTTP}
	Como se podrá ver, habían/hay \textbf{numerosas} cosas que no sabía/sé de PHP, entre esas como enviar y recibir datos por medio peticiones PHP. Por eso interesante, leer como usar la variables "\$\_SERVER" para mirar los REQUEST\_METHOD que se hacían en el servidor localmente, para poder hacer inserciones, o traer todos los clientes según fuera el caso. 
	
\end{document}